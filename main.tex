%! TeX program = lualatex
\documentclass{tesis}
\usepackage{fontspec}
\setmainfont{Times New Roman}

\newcommand{\autor}{Jeniffer Paulina Yaguana Caraguay}
\newcommand{\cedulaAutor}{1106113697}
\newcommand{\tutor}{Leonardo David Torres Valverde}
\newcommand{\lugarFechaPrelims}{Ambato, julio 2024}
\newcommand{\tema}{Prototipo de un Sistema de Monitoreo y Control para la gestión de consumo de agua en los cultivos de plantas del Vivero Michita mediante Lógica Difusa}
\newcommand{\elaboradopor}{Elaborado por: el investigador}


\begin{document}

%Start counting pages from 2 in roman numbers
\pagenumbering{roman}

%A. Paginas preliminares
\addcontentsline{toc}{chapter}{\bfseries \uppercase{PORTADA}}
\begin{titlepage}
	\begin{center}

		\includegraphics[width=50mm]{resources/images/logo-uta.png}

		\textbf{\uppercase{
				Universidad Técnica de Ambato \\
				\vspace{5mm}
				Facultad de Ingeniería en sistemas, electrónica e industrial \\
				\vspace{5mm}
				carrera de software \\
			}}

		\vspace{5mm}
		\textbf{Tema:}
		\vspace{5mm}

		\begin{tabular}[c]{>{\centering\arraybackslash}p{0.97\textwidth}}
			\toprule
			{\bfseries\MakeUppercase\tema} \\
			\bottomrule
		\end{tabular}

    \vspace{24pt}
		Trabajo de titulación modalidad Proyecto de investigación, presentado
		previo a la obtención del título de Ingeniero de Software

		\vspace{5mm}
		\textbf{\uppercase{área:}} Software\\
		\vspace{5mm}
		\textbf{\uppercase{línea de investigacion:}} Desarrollo de Software\\
		\vspace{5mm}
		\textbf{\uppercase{autor:}} \autor \\
		\vspace{5mm}
		\textbf{\uppercase{tutor:}} Ing. \tutor, Mg.\\

		\vfill
		Ambato - Ecuador\\
		febrero - 2024

	\end{center}
\end{titlepage}


\setcounter{page}{2}
% %TODO
\chapter*{aprobación del tutor}
\addcontentsline{toc}{chapter}{\bfseries \uppercase{aprobación del tutor}}

En calidad de tutor del trabajo de titulación con el tema:
{\MakeUppercase\tema},
desarrollado bajo
la modalidad Proyecto de Investigación, por el señor \autor,
estudiante de la Carrera de Software, de la Facultad de Ingeniería en Sistemas, Electrónica e Industrial, de la
Universidad Técnica de Ambato, me permito indicar que el estudiante ha sido tutorado
durante todo el desarrollo del trabajo hasta su conclusión, de acuerdo a lo dispuesto en
el Artículo 17 del Reglamento para la Titulación de Grado en la Universidad Técnica
de Ambato y el numeral 6.3 del instructivo del reglamento referido.

\begin{flushright}
	\lugarFechaPrelims.
\end{flushright}

\vspace*{5cm}
\begin{center}
	\begin{tabular}{c}
		\hbox to 7cm{\leaders\hbox to 5pt{\hss - \hss}\hfil} \\
		Ing. \tutor, Mg.\\
		TUTOR
	\end{tabular}
\end{center}

% %TODO
\chapter*{autoría}
\addcontentsline{toc}{chapter}{\bfseries \uppercase{autoría}}
El presente trabajo de titulaci\'on con el tema: {\MakeUppercase\tema} es absolutamente original, auténtico y personal y ha observado
los preceptos establecidos en la Disposición General Quinta del Reglamento para la
Titulación de Grado en la Universidad Técnica de Ambato. En tal virtud, el contenido,
efectos legales y académicos que se desprenden del mismo son de exclusiva
responsabilidad del autor.\begin{flushright}
	\lugarFechaPrelims.
\end{flushright}

\vspace*{5cm}
\begin{center}
	\begin{tabular}{c}
		\hbox to 7cm{\leaders\hbox to 5pt{\hss - \hss}\hfil} \\
		Richard Manuel Carrión Valarezo                      \\
		C.C. 0705023406                                      \\
		AUTOR
	\end{tabular}
\end{center}

% %TODO
\chapter*{derechos de autor}
\addcontentsline{toc}{chapter}{\bfseries \uppercase{derechos de autor}}
Autorizo a la Universidad Técnica de Ambato para que reproduzca total o parcialmente
este trabajo de titulación dentro de las regulaciones legales e institucionales
correspondientes. Además, cedo todos mis derechos de autor a favor de la institución
con el propósito de su difusión pública, por lo tanto, autorizo su publicación en el
repositorio virtual institucional como un documento disponible para la lectura y uso
con fines académicos e investigativos de acuerdo con la Disposición General Cuarta
del Reglamento para la Titulación de Grado en la Universidad Técnica de Ambato.
\begin{flushright}
	\lugarFechaPrelims.
\end{flushright}

\vspace*{5cm}
\begin{center}
	\begin{tabular}{c}
		\hbox to 7cm{\leaders\hbox to 5pt{\hss - \hss}\hfil} \\
		Jeniffer Paulina Yaguana Caraguay                      \\
		C.C. 1106113697                                      \\
		AUTOR
	\end{tabular}
\end{center}

% %TODO
\chapter*{aprobación del tribunal de grado}
\addcontentsline{toc}{chapter}{\bfseries\uppercase{aprobación del tribunal de grado}}
En calidad de par calificador del informe final del trabajo de titulación presentado por el señor \autor, estudiante de la Carrera de Software, de la Facultad de Ingeniería en Sistemas, Electrónica e Industrial, bajo la Modalidad Proyecto de Investigación, titulado
	{\MakeUppercase\tema}
nos permitimos informar que el trabajo ha sido revisado
y calificado de acuerdo al Artículo 19 del Reglamento para la Titulación de Grado en
la Universidad Técnica de Ambato y el numeral 6.4 del instructivo del reglamento
referido. Para cuya constancia suscribimos, conjuntamente con la señora Presidente
del Tribunal.
\begin{flushright}
	\lugarFechaPrelims.
\end{flushright}


\vspace*{4cm}
\begin{center}
	\begin{tabular}{c}
		\hbox to 7cm{\leaders\hbox to 5pt{\hss-\hss}\hfil} \\
		Ing. Elsa Pilar Urrutia Urrutia, Mg.                 \\
		PRESIDENTE DEL TRIBUNAL                              \\
	\end{tabular}
\end{center}
\vspace*{20mm}
% \begin{center}
% 	\begin{tabular}{c c}
% 		\hbox to 7cm{\leaders\hbox to 5pt{\hss - \hss}\hfil} & \hbox to 7cm{\leaders\hbox to 5pt{\hss - \hss}\hfil} \\
% 		Ing. Carlos Israel Nuñez Miranda, Mg.                & Ing. Leonardo David Torres Valverde, Mg.             \\
% 		PROFESOR CALIFICADOR                                 & PROFESOR CALIFICADOR
% 	\end{tabular}
% \end{center}

% \input{chapters/0-prelims/dedicatoria.tex}
% \newpage
\vspace*{\fill}
\begin{flushright}
	\footnotesize
	\begin{minipage}{0.5\textwidth}
		\begin{flushright}
			\phantomsection
			\uppercase{\textbf{Agradecimientos}}
		\end{flushright}
		\itshape


		\bigbreak
		\begin{flushright}
			\textbf{Jeniffer Paulina Yaguana Caraguay}
		\end{flushright}
	\end{minipage}
\end{flushright}
\vspace*{\fill}
\addcontentsline{toc}{chapter}{\bfseries AGRADECIMIENTO}


%Titulo de la tabla de contendios centrado, bold y mayusculas
\newpage
\tableofcontents
\addcontentsline{toc}{chapter}{\bfseries\uppercase{índice general de contenidos}}

\let\origaddvspace\addvspace
\newpage
\renewcommand{\addvspace}[1]{}
\listoftables
\renewcommand{\addvspace}[1]{\origaddvspace{#1}}
\addcontentsline{toc}{chapter}{\bfseries\uppercase{índice de tablas}}
\newpage
\renewcommand{\addvspace}[1]{}
\listoffigures
\renewcommand{\addvspace}[1]{\origaddvspace{#1}}
\addcontentsline{toc}{chapter}{\bfseries\uppercase{índice de figuras}}
% \newpage
% \listofappendices
% \addcontentsline{toc}{chapter}{\bfseries\uppercase{índice de anexos}}

\newpage
\listofappendixs
\addcontentsline{toc}{chapter}{\bfseries\uppercase{índice de anexos}}

\newpage
\renewcommand{\addvspace}[1]{}
\lstlistoflistings
\renewcommand{\addvspace}[1]{\origaddvspace{#1}}
\addcontentsline{toc}{chapter}{\bfseries\uppercase{índice de códigos}}

%Numeros arabicos despues de los prelims
% \chapter*{resumen ejecutivo}
\addcontentsline{toc}{chapter}{\bfseries\uppercase{resumen \uppercase{ejecutivo}}}
Las Unidades de Producción de la Universidad Técnica de Ambato son entidades
autogestionadas que generan ingresos a partir de la oferta de productos y servicios, 
representando un beneficio significativo para la universidad. Sin embargo, enfrentan desafíos en la
gestión de la información y carecen de
un medio que les permita compartir información con la dirección financiera, encargada de
recibir los pagos, para su visualización y verificación, lo cual supone un problema a la
hora de completar el proceso de entrega de productos y servicios.
\bigbreak
El presente proyecto propone la implementación de una aplicación web que establezca un 
proceso en línea para adquirir y procesar los pagos de servicios y productos, 
permitiendo a los usuarios adjuntar y verificar información relativa a los 
procesos que les corresponden. Además, se busca agilizar el registro y 
validación de datos mediante el reconocimiento óptico de caracteres (OCR) y 
procesamiento de imágenes.
\bigbreak
La aplicación web se divide en dos componentes principales: backend y frontend. En el 
backend, se utiliza Python con FastAPI para el desarrollo de la interfaz de programación de aplicaciones 
(API), integrando OCR con Pytesseract y OpenCV. 
Adicionalmente se empleó PostgreSQL como base de datos y MinIO para gestionar imágenes y archivos. 
En el frontend, se emplea React y bibliotecas adicionales.
\bigbreak
Finalmente, como resulatado se obtuvo una aplicación web que minimiza los incovenientes del proceso
actual y que, adicionalmente, descarta el uso de documentación física evitando así posibles pérdidas de
información que demoren la culminación del proceso.
\vfill
\textbf{Palabras clave:} Rappid Application Development, OCR, Tesseract, procesamiento de imagenes, aplicación web

% \newpage
\chapter*{abstract}
\addcontentsline{toc}{chapter}{\bfseries \uppercase{ABSTRACT}}
The Unidades de Producción of the Universidad Técnica de Ambato  are self-managed 
entities that generate income through the offering of products and services, 
representing a significant benefit for the university. However, they face challenges in 
information management and lack a means to share information with the finance 
department, responsible for receiving payments, for visualization and verification. This 
poses a problem when completing the process of delivering products and services.
\bigbreak
The current project proposes the implementation of a web application to establish an 
online process for acquiring and processing payments for services and products. This 
allows users to attach and verify information related to their respective processes. 
Additionally, the project aims to expedite the registration and validation of data 
through optical character recognition (OCR) and image processing.
\bigbreak
The current project proposes the implementation of a web application to establish an 
The web application is divided into two main components: backend and frontend. In the 
backend, Python with FastAPI is used to develop the Application Programming Interface 
(API), integrating OCR with Pytesseract and OpenCV. Additionally, PostgreSQL is employed 
as the database, and MinIO is used for managing images and files. In the frontend, React 
and additional libraries are utilized.
\bigbreak
In conclusion, the result is a web application that minimizes the drawbacks of the 
current process and, additionally, eliminates the need for physical documentation, 
thereby avoiding potential information losses that could delay the completion of the 
process.
\vfill
\textbf{Keywords:} Rappid Application Development, OCR, Tesseract, image processing, web application



%B. Contenidos
%Change counting to arabic numbers
\newpage
\renewcommand{\thepage}{\arabic{page}}% Roman page numbers
\setcounter{page}{1}

\chapter{MARCO TEÓRICO}

\section{Tema de investigación}
 {\MakeUppercase\tema}

\subsection{Planteamiento del problema}
\input{sections/1-tema-investigacion/planteamiento-problema.tex}
\input{sections/5-fundamentacion-teorica/antecedentes-investigativos.tex}
\input{sections/5-fundamentacion-teorica/marco-teorico.tex}

\section{Objetivo general}
\input{sections/3-objetivo-general.tex}

\section{Objetivos específicos}
\input{sections/4-objetivos-especificos.tex}

\chapter{METODOLOGÍA}
\section{Materiales}

\section{Métodos}
\input{sections/6-metodologia/modalidad-investigacion.tex}

\chapter{RESULTADOS Y DISCUSIÓN}

\chapter{CONCLUSIONES Y RECOMENDACIONES}


%C. Materiales de referencia
\bibliographystyle{biblio/IEEEtran}
\bibliography{biblio/main}
\addcontentsline{toc}{chapter}{\bfseries REFERENCIAS BIBLIOGRÁFICAS}

\chapter*{Anexos}
\addcontentsline{toc}{chapter}{\bfseries\uppercase{Anexos}}

% \addtocontents{toc}{\protect\setcounter{tocdepth}{0}}
\appendix

\renewcommand{\thetable}{\theappendixs\arabic{table}}
\renewcommand{\thefigure}{\theappendixs\arabic{figure}}
\renewcommand{\thelstlisting}{\theappendixs\arabic{lstlisting}}
\captionsetup{list=no}

% \appendixs{Guía de entrevista aplicada a los trabajadores del Vivero Michita}
% \label{app:guia-entrevista-vivero-michita}
% \begin{table}[h]
% 	\caption{Guía de entrevista aplicada a los trabajadores del Vivero Michita}\label{tab:entrevista-vivero-michita}
% 	\begin{center}
% 		\begin{tabular}[c]{p{0.3cm} p{5.5cm} p{3.5cm} p{3.5cm}}
% 			\hline
% 			\multicolumn{1}{c }{\textbf{N.}} & \multicolumn{1}{c}{\textbf{Pregunta}}                                                                                          & \multicolumn{1}{c}{\textbf{Respuesta}} & \multicolumn{1}{c}{\textbf{Observación}} \\
% 			\hline
% 			\addlinespace
% 			1                                & ¿Cuáles son los principales desafíos que enfrentan actualmente en cuanto al consumo de agua en los cultivos del vivero?        &                                        &                                          \\
% 			\addlinespace
% 			2                                & ¿Qué métodos o herramientas utilizan actualmente para monitorear y controlar el consumo de agua en los cultivos?               &                                        &                                          \\
% 			\addlinespace
% 			3                                & ¿Qué métodos de riego utilizan actualmente y con qué frecuencia?                                                               &                                        &                                          \\
% 			\addlinespace
% 			4                                & ¿Cuáles son los principales criterios que tienen en cuenta al decidir cuándo y cuánto regar los cultivos?                      &                                        &                                          \\
% 			\addlinespace
% 			5                                & ¿Cómo determinan las necesidades de agua de los diferentes tipos de plantas en el vivero?                                      &                                        &                                          \\
% 			\addlinespace
% 			6                                & ¿Qué problemas o desafíos han experimentado en el pasado con respecto al riego de los cultivos?                                &                                        &                                          \\
% 			\addlinespace
% 			7                                & ¿Existe algún tratamiento previo al agua antes de su uso en el riego? En caso afirmativo, ¿qué tipo de tratamiento se realiza? &                                        &                                          \\
% 			\addlinespace
% 			\hline
% 		\end{tabular}
% 	\end{center}
% \end{table}


\appendixs{Guía de entrevista aplicada a los trabajadores del Vivero Michita}
\label{app:guia-entrevista-vivero-michita}

\begin{longtable}{|l|p{6.5cm}|p{3.5cm}|p{3.5cm}|}
	\caption{Guía de entrevista aplicada a los trabajadores del Vivero Michita}\label{tab:entrevista-vivero-michita}                                                                                                                                                \\

	\hline \multicolumn{1}{|c|}{\textbf{N.}} & \multicolumn{1}{c|}{\textbf{Pregunta}}                                                                                         & \multicolumn{1}{c|}{\textbf{Respuesta}} & \multicolumn{1}{c|}{\textbf{Observación}} \\ \hline
	\endfirsthead

	\multicolumn{4}{c}%
	{{\bfseries \tablename\ \thetable{} -- continuación de la página anterior}}                                                                                                                                                                                     \\
	\hline \multicolumn{1}{|c|}{\textbf{N.}} & \multicolumn{1}{c|}{\textbf{Pregunta}}                                                                                         & \multicolumn{1}{c|}{\textbf{Respuesta}} & \multicolumn{1}{c|}{\textbf{Observación}} \\ \hline
	\endhead

	\hline \multicolumn{4}{|r|}{{Continua en la siguiente página}}                                                                                                                                                                                                  \\ \hline
	\endfoot

	\hline \hline
	\endlastfoot

	1                                        & ¿Cuáles son los principales desafíos que enfrentan actualmente en cuanto al consumo de agua en los cultivos del vivero?        &                                         &                                           \\
	2                                        & ¿Qué métodos o herramientas utilizan actualmente para monitorear y controlar el consumo de agua en los cultivos?               &                                         &                                           \\
	3                                        & ¿Qué métodos de riego utilizan actualmente y con qué frecuencia?                                                               &                                         &                                           \\
	4                                        & ¿Cuáles son los principales criterios que tienen en cuenta al decidir cuándo y cuánto regar los cultivos?                      &                                         &                                           \\
	5                                        & ¿Cómo determinan las necesidades de agua de los diferentes tipos de plantas en el vivero?                                      &                                         &                                           \\
	6                                        & ¿Qué problemas o desafíos han experimentado en el pasado con respecto al riego de los cultivos?                                &                                         &                                           \\
	7                                        & ¿Existe algún tratamiento previo al agua antes de su uso en el riego? En caso afirmativo, ¿qué tipo de tratamiento se realiza? &                                         &                                           \\
\end{longtable}



\section{Bibliografía}
\bibliographystyle{biblio/IEEEtran}
\bibliography{biblio/main}
\addcontentsline{toc}{chapter}{\bfseries REFERENCIAS BIBLIOGRÁFICAS}


\end{document}
