\newpage
\chapter*{abstract}
\addcontentsline{toc}{chapter}{\bfseries \uppercase{ABSTRACT}}
The Unidades de Producción of the Universidad Técnica de Ambato  are self-managed 
entities that generate income through the offering of products and services, 
representing a significant benefit for the university. However, they face challenges in 
information management and lack a means to share information with the finance 
department, responsible for receiving payments, for visualization and verification. This 
poses a problem when completing the process of delivering products and services.
\bigbreak
The current project proposes the implementation of a web application to establish an 
online process for acquiring and processing payments for services and products. This 
allows users to attach and verify information related to their respective processes. 
Additionally, the project aims to expedite the registration and validation of data 
through optical character recognition (OCR) and image processing.
\bigbreak
The current project proposes the implementation of a web application to establish an 
The web application is divided into two main components: backend and frontend. In the 
backend, Python with FastAPI is used to develop the Application Programming Interface 
(API), integrating OCR with Pytesseract and OpenCV. Additionally, PostgreSQL is employed 
as the database, and MinIO is used for managing images and files. In the frontend, React 
and additional libraries are utilized.
\bigbreak
In conclusion, the result is a web application that minimizes the drawbacks of the 
current process and, additionally, eliminates the need for physical documentation, 
thereby avoiding potential information losses that could delay the completion of the 
process.
\vfill
\textbf{Keywords:} Rappid Application Development, OCR, Tesseract, image processing, web application
