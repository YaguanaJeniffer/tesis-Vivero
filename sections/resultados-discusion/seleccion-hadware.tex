\subsection{Análisis y selección de hardware}
Para la selección de los dispositivos electrónicos que formarán parte del prototipo de sistema de monitoreo y control, se llevó a cabo un exhaustivo análisis de diversas características técnicas y tecnológicas. Se evaluaron detalladamente aspectos como la compatibilidad con el sistema propuesto, la capacidad de adquisición de datos, la precisión en las mediciones, así como la disponibilidad y el costo de cada elemento.

\subsubsection*{Sensores de temperatura y humedad}
Según se observa en la Tabla \ref{tab:comprativa_sensores_humedad_temperatura}, el sensor DHT22 sobresale por su respuesta rápida, excelente calidad y bajo consumo de energía. Su amplio rango de temperatura y humedad lo posiciona como una opción ideal para aplicaciones exigentes. A pesar de compartir características similares con el DHT11, la precisión y confiabilidad del DHT22 lo convierten en una elección sobresaliente para proyectos de monitoreo ambiental, estaciones meteorológicas y sistemas de control automático.
\begin{longtable}{|p{5.3cm}|p{3.1cm}|p{3.1cm}|p{3.1cm}|}
	\caption[Tabla comparativa de sensores de humedad y temperatura]{Tabla comparativa de sensores de humedad y temperatura\cite{tama_guerra_diseno_2018}}\label{tab:comprativa_sensores_humedad_temperatura}             \\

	\hline
	\multicolumn{1}{|c|}{\textbf{ }}         & \multicolumn{1}{|c|}{\textbf{SHT71}} & \multicolumn{1}{|c|}{\textbf{DHT11}} & \multicolumn{1}{|c|}{\textbf{DHT22}} \\
	\hline
	\endfirsthead

	\multicolumn{3}{c}%
	{{\normalfont \tablename\ \thetable{} -- continuación de la página anterior}}                                                                                 \\
	\hline
	\multicolumn{1}{|c|}{\textbf{ }}         & \multicolumn{1}{|c|}{\textbf{SHT71}} & \multicolumn{1}{|c|}{\textbf{DHT11}} & \multicolumn{1}{|c|}{\textbf{DHT22}} \\
	\hline
	\endhead
	\hline
	\endlastfoot
	Voltaje de alimentación                  & 2.4 \sim 5.5 [VDC]                   & 3 \sim 5 [VDC]                       & 3.3 \sim 6 [VDC]                     \\
	Rango de temperatura medible             & -40°C a -123.8°C                     & 0°C a +50°C                          & -40°C a +80°C                        \\
	Rango de error                           & ± 0.4 °C                             & ± 2 °C                               & ± 0.5 °C                             \\
	Número de terminales                     & 4                                    & 4                                    & 4                                    \\
	Máxima corriente de operación            & 28 uA                                & 2.5 mA                               & 1.5 mA                               \\
	Rango de humedad medible                 & 0 a 100 \%RH                         & 20 a 90 \%RH                         & 0 a 100 \%RH                         \\
	Rango de error en la medición de humedad & ±3 \%RH                              & ±4 \%RH                              & ±2 \%RH                              \\
	Tiempo de respuesta                      & 5 \sim 30 [s]                        & 1 [s]                                & 2 [s]                                \\
	Tamaño                                   & 7.42 x 4.88 x 2.5 [mm]               & 12 x 15.5 x 5.5 [mm]                 & 14 x 18 x 5.5 [mm]                   \\
\end{longtable}

\subsubsection*{Placas de Arduino}
Después de comparar las características técnicas del Arduino Uno, Arduino Leonardo y Arduino Mega, se seleccionó el Arduino Mega como la opción más adecuada para el proyecto. Esto se debe a su mayor cantidad de pines GPIO y entradas analógicas, lo que permite una mayor flexibilidad en la conexión de sensores y actuadores.
\\
\\ 
Además, el Arduino Mega cuenta con una mayor capacidad de memoria flash y SRAM, lo que lo hace más adecuado para proyectos que requieren el procesamiento de grandes cantidades de datos. Su compatibilidad con una amplia gama de shields y su recomendación como la placa más avanzada y potente de la comparación respaldan esta elección. Esta información detallada se puede encontrar en la tabla comparativa de placas de arduinos en la Tabla \ref{tab:comparativa-placas-arduino}.

\begin{longtable}{|p{4.9cm}|p{3.3cm}|p{3.3cm}|p{3.3cm}|}
	\caption[Tabla comparativa de placas de Arduino]{Tabla comparativa de placas de Arduino\cite{comparacion_nodate}}\label{tab:comparativa-placas-arduino}                                                                                                                                          \\

	\hline
	\multicolumn{1}{|c|}{\textbf{ }}            & \multicolumn{1}{|c|}{\textbf{Arduino Uno}}        & \multicolumn{1}{|c|}{\textbf{Arduino Leonardo}}                                 & \multicolumn{1}{|c|}{\textbf{Arduino Mega}}                                                                  \\
	\hline
	\endfirsthead

	\multicolumn{4}{c}%
	{{\normalfont \tablename\ \thetable{} -- continuación de la página anterior}}                                                                                                                                                                                                                    \\
	\hline
	\multicolumn{1}{|c|}{\textbf{ }}            & \multicolumn{1}{|c|}{\textbf{Arduino Uno}}        & \multicolumn{1}{|c|}{\textbf{Arduino Leonardo}}                                 & \multicolumn{1}{|c|}{\textbf{Arduino Mega}}                                                                  \\
	\hline
	\endhead
	\hline
	\multicolumn{4}{|r|}{{Continua en la siguiente página}}                                                                                                                                                                                                                                          \\
	\hline
	\endfoot
	\hline
	\endlastfoot
	Numero del elemento                         & A000066                                           & A000057                                                                         & A000067                                                                                                      \\
	Microcontrolador                            & ATmega328                                         & ATmega32u4                                                                      & ATmega2560                                                                                                   \\
	Frecuencia [Mhz]                            & 16                                                & 16                                                                              & 16                                                                                                           \\
	Interfaz                                    & USB                                               & USB                                                                             & USB                                                                                                          \\
	Voltaje de operación                        & 5                                                 & 5                                                                               & 5                                                                                                            \\
	Voltaje de entrada maximo - recomendido [V] & 12                                                & 12                                                                              & 12                                                                                                           \\
	Voltaje de entrada minimo - recomendido [V] & 7                                                 & 7                                                                               & 7                                                                                                            \\
	Opciones de alimentación                    & USB/ Ext (Auto)                                   & USB/ Ext (Auto)                                                                 & USB/ Ext (Auto)                                                                                              \\
	GPIO (PWM)                                  & 20 (6)                                            & 20 (7)                                                                          & 70 (15)                                                                                                      \\
	Entradas analogas                           & 6                                                 & 12                                                                              & 16                                                                                                           \\
	Salidas analogas                            & 0                                                 & 0                                                                               & 0                                                                                                            \\
	Memoria flash [kB]                          & 32                                                & 32                                                                              & 256                                                                                                          \\
	SRAM [kb]                                   & 2                                                 & 2                                                                               & 8                                                                                                            \\
	Puertos USB                                 & B                                                 & Micro Type B                                                                    & B/A                                                                                                          \\
	Longitud [mm]                               & 69                                                & 69                                                                              & 102                                                                                                          \\
	Ancho [mm]                                  & 53                                                & 53                                                                              & 54                                                                                                           \\
	Peso [g]                                    & 25                                                & 20                                                                              & 35                                                                                                           \\
	Compatibilidad con shields                  & Trabaja con la mayoría de placas                  & Algunas diferencias en pinouts                                                  & Algunas diferencias en pinouts                                                                               \\
	Otras características                       & -                                                 & Cliente de USB                                                                  & -                                                                                                            \\
	Soporte de comunicación inálambrica         & Sin soporte                                       & Sin soporte                                                                     & Sin soporte                                                                                                  \\
	Recomendación                               & Una buena y potenta base para empezar con Arduino & Una mejora de Arduino uno con nuevo microcontrolador y un convertidor USB menos & La más avanzada placa de está comparación - mucha fliexibilidad y potencia combinada con alta compatibilidad \\
\end{longtable}

\subsubsection*{Microcontrolador}
Basándonos en la investigación realizada por el autor en \cite{verdezoto_arauz_implementacion_2023}, coincidimos en la elección del microcontrolador ESP32 WROOM para la implementación del prototipo. Además, concordamos en que el ESP32 supera a otras opciones consideradas, como el ESP8266 y la placa Netduino-3, debido a su mayor velocidad de transmisión WLAN y su capacidad para mejorar la calidad de sonido a través de pines DAC. 
\\
\\
Consideramos que estas características son fundamentales para el adecuado funcionamiento del sistema de monitoreo y control del consumo de agua en los cultivos del vivero Michita. En resumen, respaldamos la elección del ESP32 WROOM como la opción más adecuada para cumplir con los objetivos de la investigación de manera eficiente y efectiva, alineándome con las conclusiones del autor.
\begin{longtable}{|p{3cm}|p{2.8cm}|p{2.5cm}|p{2.5cm}|p{2.8cm}|}
	\caption[Tabla comparativa de microcontroladores]{Tabla comparativa de microcontroladores\cite{verdezoto_arauz_implementacion_2023}}\label{tab:comparativa-microcontroladores}                                                                            \\

	\hline
	\multicolumn{1}{|c|}{\textbf{ }}     & \multicolumn{1}{|c|}{\textbf{Arduino Mkr 1010}} & \multicolumn{1}{|c|}{\textbf{Netduino}} & \multicolumn{1}{|c|}{\textbf{ESP8266}} & \multicolumn{1}{|c|}{\textbf{ESP32 WROOM}} \\
	\hline
	\endfirsthead

	\multicolumn{5}{c}%
	{{\normalfont \tablename\ \thetable{} -- continuación de la página anterior}}                                                                                                                                          \\
	\hline
	\multicolumn{1}{|c|}{\textbf{ }}     & \multicolumn{1}{|c|}{\textbf{Arduino Mkr 1010}} & \multicolumn{1}{|c|}{\textbf{Netduino}} & \multicolumn{1}{|c|}{\textbf{ESP8266}} & \multicolumn{1}{|c|}{\textbf{ESP32 WROOM}} \\
	\hline
	\endhead
	\hline
	\multicolumn{5}{|r|}{{Continua en la siguiente página}}                                                                                                                                                                \\
	\hline
	\endfoot
	\hline
	\endlastfoot
	Procesador CPU                       & ARM CortexM0 + SAMD21 32 (bits)                 & ARM Cortex-M4 32 (bits)                 & RISC CPU Xtensa LX106 32 (bits)        & Dual Core LX6 32(bits)                     \\
	Voltaje de alimentación              & 3.6 - 6 (V)                                     & 3.3 - 5 (V)                             & 5 (V)                                  & 5 (V)                                      \\
	Pines de entrada y salida digital    & 28                                              & 12                                      & 16                                     & 24                                         \\
	Pines de entrada analógicos          & 9                                               & 8                                       & 1                                      & 18                                         \\
	Puertos                              & Micro USB                                       & Micro USB y USB                         & Micro USB                              & Puerto Micro USB                           \\
	Dimensiones                          & 63 x 18 (mm)                                    & 25 x 62 (mm)                            & 49x26 (mm)                             & 52x28 (mm)                                 \\
	Estándar de comunicación inalámbrica & 802.15                                          & 802.11b/g/n                             & 802.11 b/g/n                           & 802.11 b/g/n                               \\
	Temperatura de operación             & 0 – 70 (°C)                                     & -40 a 125 (°C)                          & -40 a 125 (°C)                         & -40 a 80 (°C)                              \\
	Protocolos de red                    & Ethernet                                        & IPv4/TCP/IP                             & IPv4/TCP/IP                            & IPv4/TCP/UDP /HTTP                          \\
	Consumo de corriente                 & 97.5 (mA)                                       & 150 (mA)                                & 70 (mA)                                & 70 a 250 (mA)                              \\
\end{longtable}
