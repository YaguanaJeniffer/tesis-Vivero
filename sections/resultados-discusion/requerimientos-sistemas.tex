\subsection{Requerimientos del sistema}
El diseño y desarrollo del sistema de monitoreo y control para la gestión de consumo de agua en los cultivos de plantas del vivero Michita, así como la aplicación móvil asociada, deben cumplir con una serie de requisitos para garantizar su eficacia y utilidad en la gestión hídrica de los cultivos. A continuación se detallan los requerimientos del sistema:

\begin{enumerate}
	\item Facilidad de instalación y adaptabilidad
	    \begin{itemize}
		    \item El sistema debe ser fácilmente instalable en diferentes zonas del vivero, permitiendo una adaptación sencilla a las necesidades específicas de cada área de riego.
	    \end{itemize}
	\item Adquisición de datos
	    \begin{itemize}
		    \item Se requiere la capacidad de adquirir y registrar los valores de humedad del suelo y temperatura, garantizando una monitorización efectiva de las condiciones ambientales.
	    \end{itemize}
	\item Control y monitoreo del consumo de agua
	    \begin{itemize}
		    \item El sistema debe permitir un control exhaustivo y un monitoreo continuo de la gestión del consumo de agua en los cultivos.
	    \end{itemize}
	\item Programación del sistema de riego
	    \begin{itemize}
		    \item Es necesario disponer de una funcionalidad de programación para el sistema de riego, permitiendo ajustes según las necesidades específicas de cada tipo de planta y condiciones ambientales cambiantes.
	    \end{itemize}
	\item Desarrollo de aplicación móvil
	    \begin{itemize}
		    \item Se debe desarrollar una aplicación móvil intuitiva y de fácil uso que permita a los empleados monitorear y controlar el sistema, brindando acceso al historial de datos. 
	    \end{itemize}
\end{enumerate}