El agua desempeña un papel fundamental en la agricultura. Dicho esto, la agricultura es uno de los principales consumidores de este recurso hídrico. En \cite{noauthor_gestion_2018} menciona la Organización de las Naciones Unidas para la Alimentación y la Agricultura (FAO) que la actividad agrícola emplea aproximadamente el 70\% del suministro mundial de agua dulce. Es por ello, que la cantidad de agua utilizada en los campos de cultivo varía según diversos factores, como el tipo de cultivo, las condiciones climáticas,
la calidad del suelo y el método de riego empleado. Por lo tanto, la FAO \cite{noauthor_gestion_2018} indica que para aumentar la eficiencia en el uso del agua, es necesario reducir su uso entre el 25\% y el 40\% de este recurso.
\bigbreak
En \cite{temperatura_nodate} señala que las condiciones climáticas del Ecuador varían significativamente entre regiones y estaciones. En este sentido, la diversidad climática influye directamente en la disponibilidad de agua, afectando así el crecimiento adecuado de los cultivos vegetales en viveros. Es por eso que en el artículo \cite{c_estudio_2018}, se menciona que en Mulaló el 83\% de los agricultores están haciendo un uso excesivo del agua de riego, debido a que disponen de cantidades superiores a las necesarias tanto para los cultivos como para la extensión de tierra irrigada.

\bigbreak
La gestión ineficiente del agua puede disminuir la productividad agrícola en viveros. De este modo, cuando los cultivos no reciben la cantidad necesaria de agua, se reduce la eficiencia en la producción y se compromete el desarrollo deficiente de las plantas ornamentales, forestales y frutales. Por otro lado, en \cite{luz_cultivo_2014} resalta que el exceso de agua puede provocar asfixiar a las plantas y generar pudriciones en las raíces.

\bigbreak
La gestión del riego de agua en el vivero Michita se realiza de forma manual, lo que impide un control del consumo de agua por cada tipo de planta. Además, la falta de información  sobre la cantidad de agua requerida para los diferentes tipos de plantas dificulta la gestión eficiente de este recurso hídrico. 
