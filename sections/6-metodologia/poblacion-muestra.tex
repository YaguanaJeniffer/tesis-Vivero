
\subsection{Población y muestra}

Para determinar el tamaño de muestra equitativo para las tres categorías de plantas en el Vivero Michita, utilizamos una fórmula estadística con un nivel de confianza del 90\%, equivalente a una puntuación estándar de 1.645 en una distribución normal.

La fórmula utilizada para determinar el tamaño de muestra fue:

\[ n = \frac{N \times Z^2 \times S^2}{d^2 (N-1) + Z^2 \times S^2} \]

Donde:
\begin{itemize}
     \item $N$ es el tamaño total de la población, que en este caso es $210$ plantas.
     \item $Z$ es la puntuación estándar asociada con el nivel de confianza del 90\%, que es aproximadamente $1.645$.
     \item $S$ es la desviación estándar de la población, es de $1$.
     \item $d$ es el margen de error deseado, que en este caso es $0.05$.
\end{itemize}

Calculando el tamaño de muestra:

\[ n = \frac{210 \times (1.645)^2 \times 1^2}{0.05^2 \times (210-1) + (1.645)^2 \times 1^2} \]

\[ n \approx \frac{568.05}{0.5225 + 2.705} \]

\[ n \approx \frac{568.05}{3.2275} \]

\[ n \approx 176.09 \]

Redondeando este valor, obtenemos $n \approx 176$. Se estima que se requiere una muestra de alrededor de 176 experimentaciones para cada categoría. Esta cantidad se considera esencial para garantizar la validez y representatividad de los resultados al evaluar la eficiencia del sistema de riego en el Vivero Michita.
