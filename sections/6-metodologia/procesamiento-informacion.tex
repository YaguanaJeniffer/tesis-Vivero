\subsection{Procesamiento y análisis de datos}
Este apartado se enfocará en cómo se concluye con la entrevista y en como se procesarán los datos obtenidos del protocolo técnico, subrayando la importancia de una interpretación efectiva para mejorar la gestión del agua en el vivero.

\bigbreak
Basándonos en los datos recopilados durante la entrevista, podemos concluir que:
\begin{itemize}
    \item La gestión del riego en el vivero enfrenta desafíos significativos debido a la variabilidad climática, lo que destaca la necesidad de adaptar el riego a condiciones cambiantes para optimizar el uso del agua.
    \item La utilización de un método de riego por aspersión a través de mangueras se destaca como una práctica común, adaptada a las necesidades específicas de cada zona del vivero y de cada tipo de planta.
    \item Es fundamental considerar múltiples factores, como las condiciones climáticas actuales, la humedad del suelo y las características individuales de las plantas, al decidir cuándo y cuánto regar los cultivos.
    \item La observación regular y la consideración de las características individuales de las plantas son cruciales para determinar la cantidad necesaria de agua y ajustar el riego de manera precisa.
    \item Se evidencia que no se realizan tratamientos previos al agua antes de su uso en el riego, y se describe el proceso directo de aplicación de agua desde un arroyo mediante mangueras, subrayando la necesidad de evaluar la calidad del agua y su impacto en el crecimiento de las plantas.
\end{itemize}


\bigbreak
Con respecto al protocolo técnico, los datos adquiridos mediante sensores especializados serán procesados y almacenados de manera eficiente. Estos datos se almacenarán en una base de datos diseñada específicamente para este propósito. Se aplicarán procesos de normalización y limpieza para garantizar la integridad y la calidad de los datos recopilados.



