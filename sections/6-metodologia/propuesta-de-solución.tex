\subsection{Propuesta de solución}
\newpage
\input{resources/figure/Esquema-Sistema.tex}

\bigbreak
En esta investigación, se utilizarán sensores de humedad del suelo, temperatura y humedad ambiental debido a su capacidad para recopilar datos precisos y en tiempo real sobre condiciones específicas. Por lo tanto, permite ajustes y decisiones basadas en datos concretos en diversos entornos y aplicaciones. Es por ello que, el uso de un microcontrolador es ventajoso gracias a su versatilidad y capacidad para procesar datos provenientes de múltiples fuentes como los sensores. La lógica difusa de tipo mandami es necesaria por su capacidad para modelar comportamientos en entornos complejos y variables. De esta manera, es aplicable en escenarios donde la adaptación precisa a diferentes condiciones es esencial.

\bigbreak
La elección de una base de datos relacional se debe a su capacidad para organizar y relacionar datos de manera estructurada. Por efecto, esta estructura asegura la integridad y coherencia de la información, permitiendo consultas y análisis detallados en diferentes contextos. El uso de una API basada en arquitectura REST es valioso por su capacidad para facilitar la interacción entre sistemas. Dicho de otro modo, esta arquitectura permite la comunicación eficiente y estándar entre diferentes plataformas y servicios.

\bigbreak
Las aplicaciones móviles se ah seleccionado porque ofrecen acceso remoto a datos y funcionalidades, permitiendo a los usuarios interactuar desde cualquier lugar. Por consiguiente, estas aplicaciones facilitan la movilidad y la conveniencia en diversos contextos. Se opta por la arquitectura MVC porque proporciona una estructura clara y organizada para el desarrollo de aplicaciones. Es por ello que, esta separación de la lógica de negocio de la lógica de presentación simplifica la gestión de procesos permiten un desarrollo modular y mantenible en una amplia gama de aplicaciones y sistemas.

\bigbreak
De igual manera la metodología XP se elige por su capacidad para adaptarse ágilmente a los cambios que puedan surgir durante el desarrollo de un proyecto. Cabe resaltar que se enfoca en la flexibilidad, la retroalimentación continua y la adaptación a requisitos cambiantes.

\bigbreak
Se plantea el desarrollo de un prototipo de sistema de monitoreo y control para la gestión eficiente del consumo de agua en los cultivos de plantas del Vivero Michita, empleando lógica difusa. Este sistema permitirá a los responsables del vivero supervisar y regular el riego de manera precisa y adaptativa, optimizando así el uso del recurso hídrico en las áreas de cultivo. Como se muestra en la Figura \ref{fig:esquema_riego}, el diagrama de procesos del sistema es fundamental para comprender su estructura.

\bigbreak
Se emplearán diversas herramientas tecnológicas. Es por ello que se utilizará sensores de humedad del suelo, temperatura y ambiente que proporcionaran información en tiempo real sobre las condiciones del vivero. Con el apoyo de un controlador, estos datos permitirán ajustar el riego de manera precisa y eficiente. La lógica difusa se utilizará para adaptar las estrategias de riego a los cambios ambientales del vivero, optimizando así el uso del agua. La información recolectada por los sensores será analizada en una base de datos, brindando datos valiosos para mejorar el manejo del agua en el vivero. 

\bigbreak 
La API se encargar de que las aplicaciones móviles se conecten fácilmente con los datos de los sensores. Esta conexión permitirá que la información recopilada por los sensores, como datos de humedad y temperatura, sea accesible en las aplicaciones. De esta forma, los encargados del vivero podrán usar la aplicación móvil para tomar decisiones sobre el riego, basándose en estos datos en tiempo real. Este enfoque tecnológico, combinado con la lógica difusa que adapta las estrategias de riego, será clave para mejorar eficazmente el sistema de riego en el vivero.

\bigbreak 
En este sentido, este proyecto de investigación se sostiene en su viabilidad técnica, ya que es respaldada por la presencia de los recursos necesarios para su desarrollo. Además, durante mi formación académica he adquirido el conocimiento necesario para llevar a cabo la implementación del sistema informático propuesto en esta investigación. En la viabilidad operativa se busca asegurar mediante la dirección y experiencia del docente tutor, quien proporcionará la orientación necesaria en el desarrollo del proyecto. Desde el punto de vista la viabilidad económica se encuentra asegurada, debido a que la investigadora cuenta con los recursos financieros necesarios para solventar los costos relacionados con el proyecto de investigación.

% \bigbreak
% Es por ello que, la propuesta se destaca por su enfoque innovador que es el uso de la lógica difusa para el desarrollo del sistema de monitoreo y control del consumo de agua. Esta técnica de inteligencia artificial permite tomar decisiones a partir de valores imprecisos, adaptándose a las condiciones cambiantes del entorno, como las condiciones climáticas y las necesidades de las plantas.