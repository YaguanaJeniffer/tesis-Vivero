\subsection{Recolección de información}
La información para esta investigación se recolectó mediante una entrevista realizada al responsable del Vivero Michita y además un protocolo técnico.

\bigbreak
Es importante destacar que las respuestas proporcionadas a continuación no son
transcripciones exactas de la entrevista, se han reformulado para adaptarse al
contexto de la investigación en curso.
\newpage
\begin{longtable}{|p{0.5cm}|p{6.5cm}|p{5.5cm}|}
	\caption{Entrevista con el trabajador del Vivero Michita} \label{tab:entrevista-vivero-michita}                                                                                                                                                                                                                                                                                                                                                                                                                                                                                                                          \\

	\hline \multicolumn{1}{|c|}{\textbf{N. Pregunta}} & \multicolumn{1}{|c|}{\textbf{Respuesta}}                                                                                                                                                                                                                                                                                                                                      & \multicolumn{1}{|c|}{\textbf{Observación}}                                                                                                                                           \\ \hline
	\endfirsthead

	\multicolumn{3}{c}%
	{{\normalfont \tablename\ \thetable{} -- continuación de la página anterior}}                                                                                                                                                                                                                                                                                                                                                                                                                                                                                                                                            \\
	\hline \multicolumn{1}{|c|}{\textbf{N. Pregunta}} & \multicolumn{1}{|c|}{\textbf{Respuesta}}                                                                                                                                                                                                                                                                                                                                      & \multicolumn{1}{|c|}{\textbf{Observación}}                                                                                                                                           \\ \hline
	\endhead

	\hline \multicolumn{3}{|r|}{{Continua en la siguiente página}}                                                                                                                                                                                                                                                                                                                                                                                                                                                                                                                                                           \\ \hline
	\endfoot

	\hline \multicolumn{3}{|p{15cm}|}{{\textbf{Conclusion:-}  Las respuestas de la entrevista resaltan los desafíos clave en la gestión del riego en el vivero, incluida la necesidad de ajustar el suministro de agua según las condiciones climáticas y las características de las plantas. Aunque se utilizan métodos tradicionales de riego, como el uso de mangueras, se destaca la importancia de la observación regular para optimizar el proceso. Los desafíos pasados subrayan la importancia de mejorar continuamente la gestión del riego para garantizar un crecimiento saludable de las plantas en el vivero.}} \\ \hline
	\endlastfoot

	1                                                 & Los desafíos que se presentan estan relacionados con la gestión adecuada del riego en un entorno de variabilidad climática, donde las condiciones pueden oscilar entre el clima cálido y húmedo. Esta variabilidad nos obliga a ajustar nuestro riego de manera eficiente para garantizar un suministro adecuado de agua a las plantas sin desperdiciar recursos.             & Destaca la importancia de adaptar el riego a las condiciones climáticas cambiantes para optimizar el uso del agua en el vivero.                                                      \\
	2                                                 & Actualmente, utilizan un sistema de riego tradicional mediante el uso de mangueras para aplicar agua directamente a las plantas. Esto nos permite controlar manualmente el suministro de agua a cada zona del vivero según sea necesario. Además, realizamos observaciones periódicas de las condiciones del suelo y de las plantas para ajustar el riego de manera adecuada. & Enfatiza el uso de métodos tradicionales de riego y la importancia de la observación para controlar el suministro de agua.                                                           \\
	3                                                 & Utilizan el método de riego por aspersión aplicado a través de mangueras, dependiendo de las necesidades específicas de cada zona del vivero y de las características individuales de cada tipo de planta. El riego se realiza diariamente a las 8 de la mañana, aunque esta frecuencia puede variar según las condiciones climáticas y las necesidades de cada cultivo.      & Resalta el método de riego utilizado y la adaptación de la frecuencia de riego a las condiciones ambientales y las necesidades de las plantas.                                       \\
	4                                                 & Los principales criterios que tienen en cuenta para decidir cuándo y cuánto regar los cultivos incluyen las condiciones climáticas actuales, la humedad del suelo, las características individuales de cada tipo de planta y las necesidades específicas de cada zona del vivero.                                                                                             & Es necesario la importancia de tener en cuenta múltiples factores al decidir el momento y la cantidad de riego.                                                                      \\
	5                                                 & Las necesidades de agua de las diferentes plantas se determinan mediante observaciones regulares y condiciones del suelo en el vivero. Además, se tiene en cuenta las características individuales de cada tipo de planta y las recomendaciones específicas de cultivo.                                                                                                       & La importancia de la observación regular y la consideración de las características individuales de las plantas permiten determinar la cantidad necesaria de agua.                    \\
	6                                                 & Enfrentan desafíos relacionados con el exceso o la falta de agua debido a la variabilidad climática y al uso de métodos de riego inadecuados. También ah experimentado problemas de enfermedades de las plantas debido a un manejo incorrecto del riego.                                                                                                                      & Se destaca la importancia de aprender de los desafíos pasados y mejorar continuamente la gestión del riego en el vivero.                                                             \\
	7                                                 & No realizan ningún tratamiento previo al agua antes de su uso en el riego. Utilizan agua directamente de un arroyo y la aplican a las plantas mediante mangueras sin ningún proceso de tratamiento adicional.                                                                                                                                                                 & Se señala que no se realizan tratamientos previos al agua y se describe el proceso directo de aplicación de agua                                                                     \\
\end{longtable}

Además, la recolección de información se llevo a cabo a través de un protocolo técnico diseñado para la adquisición de datos del Vivero Michita, en el contexto del desarrollo de este prototipo, se ejecutará mediante una secuencia de pasos.

\subsubsection*{Sensor de Humedad del Suelo}
Se ubicará a una profundidad de 8 centímetros en áreas representativas de cada categoría de plantas del vivero Michita. Se dispondrá un sensor por cada categoría de plantas para capturar las variaciones específicas de humedad en sus respectivas áreas.

\subsubsection*{Sensor de Temperatura y Humedad Ambiental}
Se instalarán a diferentes alturas en distintos sectores para registrar las variaciones de temperatura y humedad a lo largo del día. Se colocarán de manera estratégica para abarcar áreas representativas de las distintas categorías de plantas.

\subsubsection*{Consideraciones Especiales}
Se ajustará la disposición de los sensores conforme a las variaciones topográficas y ambientales presentes. Se realizarán mediciones piloto para validar la eficacia y precisión de la ubicación de los sensores antes de su despliegue definitivo.