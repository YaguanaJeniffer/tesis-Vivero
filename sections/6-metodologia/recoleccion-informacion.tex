\subsection{Recolección de información}
La información para esta investigación se recolectó mediante una entrevista realizada al responsable del Vivero Michita y además un protocolo técnico.

\bigbreak
Es importante destacar que las respuestas proporcionadas a continuación no son
transcripciones exactas de la entrevista, se han reformulado para adaptarse al
contexto de la investigación en curso.
\newpage
\begin{table}[h]
	\caption{Guía de entrevista aplicada a los trabajadores del Vivero Michita}\label{tab:entrevista-vivero-michita}
	\begin{center}
		\begin{tabular}[c]{p{0.3cm} p{5.5cm} p{3.5cm} p{3.5cm}}
			\hline
			\multicolumn{1}{c }{\textbf{N.}} & \multicolumn{1}{c}{\textbf{Pregunta}}                                                                                          & \multicolumn{1}{c}{\textbf{Respuesta}} & \multicolumn{1}{c}{\textbf{Observación}} \\
			\hline
			\addlinespace
			1                                & ¿Cuáles son los principales desafíos que enfrentan actualmente en cuanto al consumo de agua en los cultivos del vivero?        &                                        &                                          \\
			\addlinespace
			2                                & ¿Qué métodos o herramientas utilizan actualmente para monitorear y controlar el consumo de agua en los cultivos?               &                                        &                                          \\
			\addlinespace
			3                                & ¿Qué métodos de riego utilizan actualmente y con qué frecuencia?                                                               &                                        &                                          \\
			\addlinespace
			4                                & ¿Cuáles son los principales criterios que tienen en cuenta al decidir cuándo y cuánto regar los cultivos?                      &                                        &                                          \\
			\addlinespace
			5                                & ¿Cómo determinan las necesidades de agua de los diferentes tipos de plantas en el vivero?                                      &                                        &                                          \\
			\addlinespace
			6                                & ¿Qué problemas o desafíos han experimentado en el pasado con respecto al riego de los cultivos?                                &                                        &                                          \\
			\addlinespace
			7                                & ¿Existe algún tratamiento previo al agua antes de su uso en el riego? En caso afirmativo, ¿qué tipo de tratamiento se realiza? &                                        &                                          \\
			\addlinespace
			\hline
		\end{tabular}
	\end{center}
\end{table}

Además, la recolección de información se llevo a cabo a través de un protocolo técnico diseñado para la adquisición de datos del Vivero Michita, en el contexto del desarrollo de este prototipo, se ejecutará mediante una secuencia de pasos.

\subsubsection*{Sensor de Humedad del Suelo}
Se ubicará a una profundidad de 8 centímetros en áreas representativas de cada categoría de plantas del vivero Michita. Se dispondrá un sensor por cada categoría de plantas para capturar las variaciones específicas de humedad en sus respectivas áreas.

\subsubsection*{Sensor de Temperatura y Humedad Ambiental}
Se instalarán a diferentes alturas en distintos sectores para registrar las variaciones de temperatura y humedad a lo largo del día. Se colocarán de manera estratégica para abarcar áreas representativas de las distintas categorías de plantas.

\subsubsection*{Consideraciones Especiales}
Se ajustará la disposición de los sensores conforme a las variaciones topográficas y ambientales presentes. Se realizarán mediciones piloto para validar la eficacia y precisión de la ubicación de los sensores antes de su despliegue definitivo.